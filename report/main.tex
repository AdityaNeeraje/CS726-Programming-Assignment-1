\documentclass{article}
\usepackage{amsmath}
\usepackage{amssymb}
\usepackage{amsthm}
\usepackage{listings}
\usepackage{float}
\usepackage{graphicx}
\usepackage[includefoot,bottom=12pt]{geometry}
\usepackage{hyperref}
\usepackage{subcaption}
\usepackage{fancyhdr}
\usepackage{bbm}
\usepackage{tikz}

\geometry{left=1.25in}

\pagestyle{fancy}
% \fancyhead{}
% \fancyfoot{}

\makeatletter
\newcommand*{\centernot}{%
  \mathpalette\@centernot
}
\def\@centernot#1#2{%
  \mathrel{%
    \rlap{%
      \settowidth\dimen@{$\m@th#1{#2}$}%
      \kern.5\dimen@
      \settowidth\dimen@{$\m@th#1=$}%
      \kern-.5\dimen@
      $\m@th#1\not$%
    }%
    {#2}%
  }%
}
\makeatother

\newcommand{\independent}{\perp\mkern-9.5mu\perp}
\newcommand{\notindependent}{\centernot{\independent}}

\fancyhead[R]{Neeraje - 23B0940}
\fancyhead[L]{\leftmark}

\renewcommand{\sectionmark}[1]{\markboth{#1}{}}

\title{AML Assignment-1}

\author{Aditya Neeraje}

\date{\today}

\begin{document}

\maketitle
\tableofcontents

\pagenumbering{gobble}

\newpage

\pagenumbering{arabic}
\section{Generating and Triangulating the Graph}
Some of the preprocessing starts when I initialize the Inference class itself. I create an instance of the Graph class, and add edges in it according to the cliques in the given graph. I also note how many cliques each vertex is part of, which comes in handy later to speed up finding simplicial vertices.\\
Here, to avoid incorrect counting of the number of cliques a node is in if one of the input cliques is a subset of another, I sort the cliques in increasing order of sizes, and consider a clique only if it is maximal (and consider only one clique if there are multiple instances of the exact same clique in the graph).\\
I then store the graph's state (adjacency list, number of cliques and number of neighbours, etc\.) so that I can revert to the original state after triangulating using one heuristic (say, min-neighbours), before triangulating based on another heuristic (say, min-fill).\\
\\
Then, when \texttt{triangulate\_and\_get\_cliques} is called, the heuristic is used to find a close-to-optimal triangulation of the graph. Since the implementation of all heuristics is fundamentally very similar, I will here only elaborate on the implementation of min-neighbours. First, I separate the corner case of nodes with no neighbours, and start maintaining a list of simplicial nodes. While there are nodes yet to be included in the ordering, I search for simplicial nodes and add them to the ordering, and implicity delete it from the graph by setting its number of neighbours to 0 (which is the check I use later to see if a node is deleted). I keep updating the cliques by removing nodes which have been eliminated, until I no longer find any simplicial vertex. Now, we choose the vertex with the fewest neighbours, and create a new clique with it and its neighbours. I also delete all those cliques which earlier contained this vertex. Now, if the addition of this clique causes a previous clique to not remain maximal, I delete that clique. I keep repeating this process until all nodes are eliminated.

\section{Junction Tree Creation}

\end{document}